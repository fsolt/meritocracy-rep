% Build on the outline below.  Cite by using \citep{Author2016} to add a
% parenthetical citation; use \citet{Author2016} to get a textual cite like
% this: Author (2016). 

\section{Multiply Impute Missing Data}

Summary Paragraph\par

In political science study, a common way to contaminate the data and undermine the validity is missing data growing non-response in Pew survey (Curtain, et.al, 2005;);\par

Reason: poor survey design(literature needed); private information (income;  ); sensitive question—influence of social desiablity (underreported abortion of black: Jagannathan, 2001; nonrespond and overreponse in female president question; Streb, 2008; nonresponse in religious: Hadway et.al, 1993); survey forms (Chang RDDv.s. Internet; 2009; Amazon-Turk, Berinsky, 2012)\par

To deal with missing data: first identify missing mechanisms: (1) missing completely at random (MCAR): the probability of missingness is the same for all unit; that is, each survey respondent choose not to answer the question based on on coin flip (2)missing at random (MAR), and nonignorable or (King et. al, 2001): the probability whether the survey question is answered may depend on the other factor which are observable in the data: for example, an independents are more tend to decline to answer partisan identification question. (3)nonignorable (NI): because of the unobserved value of the missing response: high income people tend to conceal their real income, and other variables in the data cannot predict which respondent have high income.\par

Strategy: to all: 1. listwise deletion or complete-case analysis; +  Available case study  (use the distribution of other observable variables)\par

Disadvantages: (1) lead to biased estimates, especially when missing values differ systematically from the completed observed cases; (2) relatively, the standard error can be sensitive due to the original sample size and the deleted case size; (3) in ACS, may lead to omission of a variable hat is necessary to satisfying the assumptions necessary for desired casual inference.\par

2. Non-response weighting: need to add literatures (for MAR and MCAR)\par

3. Simple imputation: With high certainty: (1 )mean, (2)last value carried forward;\par

With uncertainty: (3) using information from related observation (for example, in GSS, using reported occupation types and its mean annual salary to infer income; or in SIS, use reported working months to infer income)\par
 
Disadvantages:  (1) mean, last value, or other singly imputed method bias, especially when the size of missing observation is relatively large; distorting the actual distribution (Gelman, 2006)\par

(2) Inferred from other related observation: may not to impute all missing data\par

4. Random Imputation with a single variable or multiple variables\par

5. Multiple Imputation:\par

Definition: imputing missing values for each missing case with different imputations to reflect uncertainty levels, and creating a complete data set. (King.et.al, 2001). For example, if assume the missing data of a specific variable is MAR, indicating other observable variables can infer useful information to predict the missing cases,  conditional on whichever imputation model adopted.\par

(1)Multivariate regression:\par
a. use continuous model to impute missing discrete response.Example: modeling the data as continuous (transfer discrete value into continuous), and imputing continuous values (Gelman and King, 1998) \par

(2)

Specific to Newman’s paper\par
church attendance---all missing are simply assigned ``once or twice a month"

income, a variable of interest, is missing for over 10\% of the sample, but values are mysteriously single-imputed (where did these values come from? they aren't meaningful--they fall between categories)

ideology, partyid also single-imputed, it seems

Church attendance: MAR

income: MAR

ideology: MAR

Thus, mulitple imputation is possible and necessary

missing data should be multiply imputed \citep[e.g.,][]{King2001}




% Build on the outline below.  Cite by using \citep{Author2016} to add a
% parenthetical citation; use \citet{Author2016} to get a textual cite like
% this: Author (2016). 

\section{Work in Public}

Researching transparency is always an important criterion for scientific researches, including qualitative and quantitative political scientist studies \citep{Appadurai2000,Denzin2009}. One way to match this standard is to open every data managing and estimating step in public. It may sound time consuming, unsafe, and unnecessary, especially in the views of researchers who already provide ``replication file'' (\cite[e.g., ][]{Newman2015}). But, as shown in the previous section, the actual problem is the offered ``replication file'' does not always work (\cite[e.g., again, ][]{Newman2015}). In this case, other researchers can go directly to the publicly accessed documents and files which the researcher used to conduct the analyses to check what was wrong in the replication file. 

% Let's keep Newman2015 separate: i.e., we need to talk in general before getting to our specific example.  https://politicalsciencereplication.wordpress.com/2015/05/04/leading-journal-verifies-articles-before-publication-so-far-all-replications-failed/ and http://www.federalreserve.gov/econresdata/feds/2015/files/2015083pap.pdf may be useful references re whether replication files actually replicate

Moreover, the contemporary computer and internet technologies also make working in public very easy and safely. Taking GitHub as an example,\footnote{There are alternative agents, such as SVN and Gitlab, offering similar services, although there might be slight difference in operations and compatible software, and most of them are free to register and use.} what requires the researcher to do is simply to spend 10 second to build a repository and push the new committed file (viz. the updated file) after every important change in the analysis, as we did for \href{https://github.com/fsolt/meritocracy-rep}{this project}. At the same time, the study per se is also safe as long as the paper and data is not accessed, which is actually also not required before the study is published. 

Another way to do publicly study is to preregister the research. It is in the same line as pushing staged analyses onto GitHub, but under a more specific supervision of the academia of political science. The preregistration asks researchers to publish their research plan prior to conducting the analyses. The purpose of this is mainly to avoid the result-oriented researches in which the researcher manipulate the data or change the theory and hypotheses based on the empirical results they have.\footnote{More discussions about preregistration were in the \href{http://pan.oxfordjournals.org/content/21/1.toc}{``Symposium on Research Registration''}, of Winter 2013 issue of \textit{Political Analysis}}



% Build on the outline below.  Cite by using \citep{Author2016} to add a
% parenthetical citation; use \citet{Author2016} to get a textual cite like
% this: Author (2016). 

\section{Work in Public}

Researching transparency is always an important criterion for scientific researches, including qualitative and quantitative political scientist studies \citep{Appadurai2000,Denzin2009}. One way to achieve it is to open every data managing and estimating step in public. It may sound time consuming, unsafe, or unnecessary, especially in the views of researchers who already provide ``replication file.'' Nevertheless, the issue is not only that the replication files are not always replicable in social scientific studies \citep{Chang2015,Jacoby2015,OpenScienceCollaboration2015}. But also for the readers who intend to really replicate some part or the entire analysis for further studies, the ``replication file'' in many cases may not be adequate to fully track the decisions and manipulations the authors made. In such cases, a public-accessed file provides another chance to review what exactly happened during the analyses and with what decisions and movements the conclusion was conducted. 

Nowadays, the contemporary computer and internet technologies offers easy and safe ways for researchers to work in public. Taking GitHub as an example,\footnote{There are alternative agents, such as SVN and Gitlab, offering similar services, although there might be slight difference in operations and compatible software, and most of them are free to register and use.} what requires the researcher to do is simply to spend 5 second to build a repository for later pushing updated committed file after changes, as we did for \href{https://github.com/fsolt/meritocracy-rep}{this project}. This way also benefit the researchers by recording every step during the analysis, in order to later review, restore, and create replicate files. At the same time, the study per se is also safe as long as the paper and data is not accessed. 

Another way to do publicly study is to preregister the research. It is in the same line as pushing staged analyses onto GitHub, but under a more specific supervision of the academia. The preregistration asks researchers to publish their research plan prior to conducting the analyses. The purpose of this is mainly to avoid the result-oriented researches in which the researcher manipulate the data or change the theory and hypotheses based on the empirical results they have.\footnote{More discussions about preregistration were in the \href{http://pan.oxfordjournals.org/content/21/1.toc}{``Symposium on Research Registration''}, of Winter 2013 issue of \textit{Political Analysis}}



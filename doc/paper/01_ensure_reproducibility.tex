% Build on the outline below.  Cite by using \citep{Author2016} to add a
% parenthetical citation; use \citet{Author2016} to get a textual cite like
% this: Author (2016). 

\section{Ensure Reproducibility}

Ensure Reproducibility

What is reproducibility? Reproducibility means that Researcher B obtains exactly the same results that were originally reported by Researcher A (e.g. the author of that paper) from A’s data when following the same methodology (Brunswik 1955; Asendorpf et al. 2013). 
Also, even for replicability, this paper is problematic. Replicability means that the finding can be obtained with other random samples drawn from a different time point or a different situation. As we will show as follows, they employed one survey that can strengthen their arguments and did not try to test whether their findings can be generalized to other time points and other situations. 

Why reproducibility is important.
How the present social science could develop so far based on the replication work. Examples.
Using multiple measures for the dependent variable. 

Reproducibility as bare minimum for replication; DA-RT APSA guidelines

script all work

packrat and checkpoint packages in R; version command in Stata

quote \citet{Newman2015} replication materials

Table 1 and 2 cannot be reproduced exactly

Table 3 cannot be reproduced at all: more parameters than observations
